Os sistemas de compartilhamento de bicicletas vêm se popularizando no mundo inteiro, visto os benefícios que são alcançados com a implementação deles nos grandes centros urbanos, como a redução de congestionamentos de automóveis e a melhoria da qualidade de vida das pessoas. Desta forma, é proveitoso estudar os problemas ligados a tais sistemas, como a questão de distribuição das bicicletas pelas estações e também o melhor modo de como isso deve ser feito. Alternativas serão estudadas e sugeridas neste documento com o intuito de prover resultados aceitáveis que ajudem na implantação de tais sistemas. Essas alternativas serão embasadas por modelos matemáticos e meta-heurísticas de otimização combinatória.