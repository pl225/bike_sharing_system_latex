\chapter{Revisão da literatura}\label{chp:LABEL_CHP_2}

\par Neste capítulo, o leitor encontrará sínteses de trabalhos anteriores de autores que estudaram e desenvolveram modelos, exatos ou não, para enfrentar problemas, que aqui foram agrupados se fossem de alguma forma relacionados ao BSS, e tentar resolvê-los. É importante observar que o problema de rebalanceamento de bicicletas nos sistemas de compartilhamento pode ser visto como um problema de otimização combinatória. As diversas categorizações deste problema com relação ao tamanho da frota de veículos, se são permitidas múltiplas visitas às estações e se é permitida a técnica de preempção, foram identificadas como pertencentes ao conjunto dos problemas $\mathcal{NP}$-Hard.

\section{Hernández-Pérez e Salazar-González, 2004}\label{sec:LABEL_CHP_2_SEC_A}

Neste primeiro artigo, os autores não estudaram propriamente o BSS, porém, ele é citado em diversos trabalhos que realmente o analisaram, visto que os conceitos por ele apresentados são muito semelhantes aos do BSS. O objeto de estudo deles foi o Problema do caixeiro viajante de entrega e coleta de um produto, \textit{one-commodity pickup-and-delivery traveling salesman problem - 1-PDTSP}, em inglês, que está intrisecamente ligado ao problema do caixeiro viajante original, \textit{traveling salesman problem - TSP}, em inglês. Por alto, ele possui as seguintes características: cada estação possui uma demanda de entrega ou de coleta de um mesmo produto, o veículo tem uma capacidade máxima de carga determinada, cada estação só pode ser visitada uma vez (em outras palavras, temos um circuito hamiltoniano) e existe uma estação especial chamada de depósito. O objetivo seria encontrar uma rota de custo mínimo, iniciando e finalizando no depósito, que compreendesse todas as estações com ou sem demanda. Ao final, quando o veículo visitasse todas as estações da rota, coletando ou entregando o produto, não mais haveria demanda em nenhuma delas. Cada visita não poderia infringir as capacidades máxima e mínima de carga do veículo. Os estudiosos deste problema construíram um modelo de programação linear inteira e desenvolveram um algoritmo \textit{branch-and-cut} para encontrar uma solução ótima para o modelo proposto, que suportava, em tempo hábil, instâncias do problema com até 60 estações. Deste artigo, diversos autores, inclusive os que serão citados nesse trabalho, utilizam as istâncias descritas nele para a comparação de diversos algoritmos desenvolvidos para resolver não só o 1-PDTSP, como também variações do BSS e do TSP.

\section{Hernández-Pérez, Salazar-González, Rodrígues-Martín, 2008}\label{sec:LABEL_CHP_2_SEC_B}

Os autores do trabalho anterior nesse artigo procuraram outro meio de atacar o problema 1-PDTSP, desta vez com uma tática inexata, utilizando as heurísticas GRASP, para construção de uma solução inicial, e a VND, na itensificação pela busca de outras soluções a partir daquela inicial. Eles compararam essa técnica com a do artigo anterior e os testes realizados com as mesmas instâncias mostraram que a metodologia heurística empregada produziu resultados melhores que as da tática exata. Eles também mostraram que a averiguação da viabilidade de uma solução poderia ser calculada em tempo linear em relação ao número de estações presentes na solução. As vizinhanças utilizadas no procedimento VND foram variações dos operadores 2-opt e 3-opt de troca de arestas. Outras duas estruturas de vizinhança foram definidas na fase que os autores nomearam como pós-otimização, na tentativa de melhorar a melhor solução até então escolhida.

\section{Chemla, Meunier, Calvo, 2013}\label{sec:LABEL_CHP_2_SEC_C}

O trabalho dos autores desta seção é o primeiro da lista que já estuda o BSS, abordando o rebalanceamento estático e a permissão da técnica de preempção. Os autores desenvolveram um modelo exato, que logo se mostrou intratável. Eles então relaxaram o modelo, chegando a um problema de programação linear inteira com um número exponencial de restrições, para o qual foi proposto um algoritmo \textit{branch-and-cut}, como no artigo de \citet{art:REF_ART_3}, para encontrar um limite inferior em relação à solução ótima do problema original. Para traçar os limites superiores do problema, os autores empregaram a busca tabu. Ao longo do trabalho, os autores provam as várias proposições utilizadas por eles no desenvolvimento do algoritmo final, entre as quais vale citar a decisão em tempo polinomial se uma sequência de vértices induz uma solução viável ou não e também, no mesmo tempo algorítmico, as operações possíveis que levam o sistema ao estado mais próximo do objetivo, dada a mesma entrada da proposição anterior.

\section{Paes, Subramanian, Ochi, 2010}\label{sec:LABEL_CHP_2_SEC_D}

Este artigo se trata de outra maneira de lidar com o 1-PDTSP introduzido pelos autores da seção \ref{sec:LABEL_CHP_2_SEC_A} e utiliza heurísticas diferentes das apresentadas na seção \ref{sec:LABEL_CHP_2_SEC_B}, que são o GRASP, o ILS e o RVND. Além dessas técnicas, os autores ainda implementaram um pré-processamento das instâncias do problema, utilizando 3 restrições que impedem que o espaço de busca inteiro de soluções seja utilizado num primeiro momento na busca por soluções viáveis. O resto desse espaço só é utilizado quando não é possível encontrar um caminho viável utilizando apenas a melhor parte dele. Um exemplo de restrição é a não consideração dos arcos entre as estações que possuem um custo maior que o custo médio entre todos eles. Os autores também desenvolveram uma função de avaliação de inserção de uma estação num caminho corrente que leva em conta o custo entre os arcos e a taxa de violação da carga do veículo. Eles aplicaram 5 estruturas de vizinhança no algortimo RVND e apenas o procedimento \textit{double-bridge} na fase de perturbação. O experimentos realizados, comparados àqueles do trabalho da seção \ref{sec:LABEL_CHP_2_SEC_B}, mostraram que a técnica empregrada obteve desempenho satisfatório nas instâncias pequenas em relação ao tempo, e, quando instâncias de tamanho maior foram utilizadas, ela melhorou as soluções ótimas até então conhecidas na literatura. 

\section{Cruz, Subramanian, Bruck, Iori, 2016}\label{sec:LABEL_CHP_2_SEC_E}

Os autores do artigo desta seção estudaram a mesma variação do BSS da seção \ref{sec:LABEL_CHP_2_SEC_C}, utilizando outras técnicas para trabalhá-lo e analisá-lo. Empregando um modelo inexato, utilizando as herísticas ILS e RVND aliadas aos conceitos desenvolvidos no trabalho da seção já citada, os autores reportaram melhores resultados para o problema. Eles utilizaram 6 estruturas de vizinhança na fase de busca local com o RVND, e 4 procedimentos na fase de perturbação da solução. Além disso, eles perceberam que o processo proposto por \citet{art:REF_ART_2} para verificar em tempo polinomial se uma sequência de vértices induz uma solução viável é custoso, e que em futuros trabalhos pode ser trabalhado e aperfeiçoado. Os autores também analisaram o peso das combinações das perturbações utilizadas no tempo final do algoritmo.

\section{Cruz, Subramanian, Bruck, Iori, 2016}\label{sec:LABEL_CHP_2_SEC_F}

Os mesmos autores do artigo da seção anterior também estudaram o BSS com rebalanceamento estático sem preempção. Eles utilizaram as heurísticas ILS e RVND, contudo, a fase de perturbação da solução ainda emprega a heurística \textit{simulated annealing}, a qual permite que uma solução pior que a já encontrada seja aceita. Os autores deixaram de usar as vizinhas do trabalho anterior que permitiam a técnica de preempção diminuindo para 2 os procedimentos utilizados na perturbação de soluções. Porém, um diferencial em relação ao trabalho predecessor é a utlização de estruturas de dados que permite determinar, em tempo constante, se um movimento numa solução já construída produzirá uma solução viável ou não. Essas estruturas foram primariamente definidas no trabalho que será sumarizado na próxima seção.

\section{Bulhões, Subramanian, Erd\u{o}gan, Laporte, 2016}\label{sec:LABEL_CHP_2_SEC_G}

Diferentemente dos trabalhos anteriores, esta variação estuda o BSS com não apenas um veículo percorrendo as estações, mas vários de igual capacidade. Os autores apresentaram a heurística ILS acompanhada do RVND, como nos trabalhos anteriores, mas também desenvolveram uma formulação de programação linear e um algoritmo BC associado. Nos trabalhos anteriores, o objetivo era encontrar uma rota de custo mínimo, enquanto que neste o objetivo é encontrar uma configuração de rotas, visto que haverá uma rota para cada veículo saindo e retornando ao depósito, de custo mínimo. Algo a ser citado também é a proibição de uma estação ser visitada por veículos diferentes durante o processo. Como dito anteriormente, o trabalho da seção anterior utiliza as estruturas de dados deste artigo para verificar a viabilidade de um movimento em tempo constante sobre uma sequência de vértices. Tais estruturas são de extrema importância, por conta da existência de estruturas de vizinhança numa e entre rotas. Os mecanismos de perturbação também podem ser aplicados entre rotas. Ao final, os autores comparam os resultados entre os métodos heurístico e exato, mostrando que o primeiro teve um desempenho consideravelmente melhor em relação ao segundo.