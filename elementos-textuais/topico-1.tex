\chapter{Introdução}\label{chp:LABEL_CHP_1}

Com o aumento do número de veículos em circulação nos centros urbanos, principalmente carros pessoais e transportes públicos, vários problemas surgiram. O mais marcante deles, e de maior destaque na vida corriqueira das pessoas, é o trânsito \citep{site:1}. Muitas horas são perdidas \citep{site:2} dentro de um espaço pequeno, por vezes repleto de outros indivíduos, que juntos batalham contra o calor e o estresse intenso. E, assim, o dia de trabalho, estudo ou afins, já se inicia de forma árdua e complicada. 

\par Um problema emergente do tráfego intenso de automóveis é a poluição do ar \citep{site:3}. Um agravante do efeito estufa, chuvas ácidas e da sensação térmica \citep{site:5}, a emissão descontrolada de gás carbônico pode ainda se oferecer como um ponto de partida para diversos problemas de saúde como doenças respiratórias e cardiovasculares \citep{site:4}. 

\par Outra questão, que não deve deixar de ser enunciada, é a poluição sonora e/ou visual \citep{site:6} que pode surgir como resultado do grande número de carros em uma cidade. A primeira pode ajudar na perda gradativa de audição dos pedestres e de todas as pessoas que vivem próximas do intenso barulho e ruído de motores automobilísticos, assim como prejudicar o sono matinal de cidadãos que habitam próximos a rodovias, ao passo que a segunda, ainda que não causadora de problemas de saúde, acaba prejudicando a beleza da cidade e tornando-a difícil de ser apreciada. 

\par Assim, várias soluções são propostas e tendem a ser implementadas. A persuação dos cidadãos pela troca do transporte individual pelo público vem a calhar quando o objetivo é diminuir a circulação de carros \citep{site:7}. Porém, é notável que o sistema de transporte urbano não seria inteiramente capaz de comportar os muitos passageiros a mais que começariam a utilizá-lo em diversos centros urbanos de países ainda em desenvolvimento \citep{site:8}. A intensa lotação dos transportes, a baixa qualidade dos carros da frota, os atrasos nos horários de saída dos veículos e o alto custo da passagem têm um grande poder de afugentar os novos usuários. Felizmente, outra medida para o problema seria o aumento da capacidade de carros das vias já construídas e a construção de novas. Em contrapartida, o alto custo das obras necessárias e o tempo de conclusão delas podem atrasar a percepção dos benefícios acarretados por elas pelos usuários. Além disso, há o risco dessas novas vias públicas não serem amplamente utilizadas logo após a finalização de suas obras, como está sendo o caso do Arco Metropolitano, uma grande rodovia que interliga várias cidades ao norte da Baixada Fluminense, que possui como um alvo o desafogamento de uma das principais avenidas da capital do estado do Rio de Janeiro, no Brasil, a Avenida Brasil. Até o presente momento, o tráfego nessa via é intenso, não somente pela não utilização do Arco, mas também pelas obras em curso nessa região. Além disso, estudos revelam que apenas construir novas vias pode não resolver inteiramente o problema \citep{site:9}.

\par Frente a esses diversos problemas, uma solução implementada em mais de 400 cidades mundo afora \citep{book:1}, os sistemas de bicicletas compartilhadas, \textit{Bike-sharing system - BSS}, em inglês, vêm ganhando cada vez mais força. Apesar de parecer super moderna e visionária, esta ideia não é tão nova assim. O primeiro sistema de compartilhamento de bicicletas documentado data de 1965, na Europa, em Amsterdã, Holanda. Os maiores sistemas se encontram nas cidades de Hangzhou e Xangai, ambas na China; Paris, França; Londres, na Inglaterra; e na capital dos Estados Unidos, Washington, D.C. Essencialmente, esse projeto é conceitualmente simples: os ciclistas recolhem uma bicicleta num local, usam-nas, e as entregam em outro local quando acabam de utilizá-las. Este tipo de sistema traz consigo as vantagens de introduzir um novo tipo de transporte não poluente nos centros urbanos e consideravelmente mais barato; estimula a população no combate ao sedentarismo fomentando-a na busca por um estilo de vida mais saudável; reduz os grandes engarrafamentos nos centros urbanos; e ainda promove a humanização do espaço urbano e o senso de responsabilidade social e ambiental nos cidadãos das grandes cidades. No Brasil, já existem diversos programas desse tipo como o +Bike, no Distrito Federal, que segundo dados do próprio, contava com mais de 160.000 usuários cadastrados até outubro de 2017, podendo isso ser confirmado em \url{http://www.maisbikecompartilhada.com.br}. 

\par Algumas variáveis que precisam ser levadas em consideração nesses projetos são a capacidade capacidade máxima de bicicletas nos pontos de entrega e recolhimento, comumente nomeados de estações, o número delas disponíveis para serem coletadas e os espaços livres destinados para o retorno das que não estão mais em uso \citep{art:REF_ART_1}. Assim, não é difícil chegar a conclusão de que em algum momento uma estação pode não possuir vagas nem bicicletas para serem utilizadas por um ciclista. Portanto, uma desvantagem desses sistemas é que, em algum momento, haverá a necessidade do procedimento chamado rebalanceamento de estações, que consiste da redistribuição de bicicletas pelas estações, a fim de que o sistema possa ser utilizado novamente. A priori, não existe um momento definido para que isso ocorra. A informação necessária para avaliar se o rebalanceamento deve acontecer advém das várias observações de inúmeras execuções diárias do programa, para quantificar um número ideal de bicicletas em cada estação a partir do grau de utilização delas. Em posse de tal conhecimento, torna-se concebível prever em que estado de desequilíbrio de distribuição de bicicletas o rebalanceamento deve ser iniciado. \par Algo que caracteriza o procedimento de rebalanceamento é o momento em que ele é feito. Se acionado durante a execução do sistema, ele é denominado dinâmico e, estático, em caso contrário. Num primeiro momento, pode ser estranho o segundo modo de rebalanceamento, visto que o sistema estaria inacessível aos seus usuários durante o processo, entretanto, diversos autores, como \citep{art:REF_ART_2}, mostram que esse método pode ser executado durante a noite, quando alguns desses sistemas são fechados. Este trâmite pode ser realizado por um ou vários veículos, com capacidade determinada, que visitam as estações coletando ou entregando bicicletas. O uso de veículos leva a outras questões tais como o caminho a ser percorrido para visitar todas as estações, visto que serão gastos recursos com compra de combustíveis, contratação de motoristas e  despezas naturais oriundas da manutenção de cada automóvel da frota utilizada. \par Tendo em vista os diversos benefícios propiciados pelo BSS, é extremamente válido o estudo do seu problema de rebalanceamento, a fim de que esse sistema seja aprimorado e mais amplamente utilizado pelas pessoas, para então compartilharmos de uma melhor qualidade de vida e vivermos mais harmoniosamente com o ambiente que nos rodeia.

\section{Definição formal do problema}\label{sec:LABEL_CHP_1_SEC_A}
Agora que já temos um norte sobre o que será tratado neste trabalho, partamos agora para uma definição matemática, a qual será usada ao longo deste estudo. Tal definição é a mesma utilizada em \citep{art:REF_ART_1}.

\par Já se sabe que o sistema de compartilhamente de bicicletas, eventualmente, necessitará de um rebalanceamento. Foi visto também que tal procedimento pode ser realizado por um conjunto de veículos de capacidade limitada, digamos, por exemplo, um caminhão. Desta forma, um ou mais caminhões podem visitar as estações coletando ou entregando bicicletas. Observa-se que a capacidade do caminhão é um fator muito importante, visto que um caminhão, ao chegar num ponto de visita, pode não ter capacidade suficiente para coletar as bicicletas ou um número suficiente delas para entregá-las. Decorre-se disso que mais de uma visita pode ser feita a fim de levar uma estação ao seu número de bicicletas ideal. O ato de balancear uma estação é chamado de fechamento. Neste trabalho, será adotado o rebalanceamento estático com apenas um veículo no sistema. \par Outro fato que é muito estudado é o uso ou não de operações temporárias, chamadas também de operações preemptivas. Pelo que foi dito até agora, mesmo que sejam necessárias várias visitas a uma estação para fechá-la, uma operação sempre diminuiria a sua demanda, seja de entrega ou de coleta. Uma operação temporária pode piorar a situação de uma estação, fazendo a operação inversa à necessária para balanceá-la e, posteriormente, visitá-la outras vezes para eliminar as suas necessidades. Neste trabalho, esse tipo de operação não é permitida. \par Foi enunciado também que o uso de caminhões (ou qualquer outro meio transportador) não é gratuito. Vários custos estão envolvidos. Logo, é preferível que uma rota de visita entre as estações tenha o menor tamanho possível, levando também a um tempo menor de viagem.
\par A partir das informações já providas, o modelo de resolução do problema é descrito da forma que se segue. Seja \begin{math} n \end{math} o número de estações, \begin{math} V = \{0, 1, ..., n\} \end{math} um conjunto de vértices, no qual cada elemento representa uma estação de uma instância do problema, \begin{math} A = V \times V \end{math} um conjunto de arcos e \begin{math} G = (V, A) \end{math} um digrafo completo, em que o vértice zero representa uma estação especial, o depósito, que é o ponto de partida e de chegada para iniciar o percorrimento de todas as estações que possuam demanda por bicicletas. Para cada arco \begin{math} a_{i, j} \in A \end{math} é atribuído um custo estritamente positivo \(c_a\), satisfazendo a desigualdade triangular \( c_{i,j} + c_{j, k} \geq c_{i,k}, \forall{i, j, k} \in V \). Também assume-se que \( c_{i,j} + c_{j, i}, \forall{i, j} \in V \). 

\par Para todo \(i \in V \), existe um número \( p_i \) de bicicletas em \( i \) antes da inicialização do serviço e um número ideal de bicicletas \( p_i^{'} \). A demanda de \( i \) é \( d_i = p_i^{'} - p_i  \), que, se maior que zero, indica uma necessidade de entrega de bicicletas, e de coleta em caso contrário. Se \( p_i^{'} = p_i \), tal estação não tem demanda e uma visita a ela é opcional. Seja um número estritamente positivo \( Q \) representando a capacidade do veículo. Seja uma sequência de vértices, que se inicia e termina no depósito (um circuito). A cada visita a um vértice, existe uma operação \( g_i \) associada, que é menor que zero, caso tenha sido executada uma entrega e maior que zero, se tiver sido feita uma coleta. \par O objetivo é prover uma rota de custo mínimo que se inicie e termine no depósito; a cada visita a uma estação, as capacidades mínima, zero, e máxima, \(Q\), do veículo foram respeitadas, e ao final, todas as estações que possuíam demanda foram fechadas, tendo sido visitadas ao menos uma vez. Abaixo é mostrado um exemplo de instância com \( n = 8 \) e \( Q = 8 \). A rota tem o caminho \( l = <0, 7, 6, 4, 5, 2, 8, 1, 0>\) associado, e a sequência de operações é \( g = <0, +2, +6, -4, +4, -1, -6, -1>\). Além de representar a sequência de operações realizadas, também pode-se apresentar a quantidade de bicicletas no veículo a cada visita a uma estação durante o trajeto, assim como a quantidade de vagas livres nele nas mesmas condições. Para a primeira representação, conforme o exemplo, teríamos \(h = <0, 2, 8, 4, 8, 7, 1, 0>\). Enquanto que para a segunda, o que há é \( i = <8, 6, 0, 4, 0, 1, 7, 8>\). 

\begin{figure}[hb]
    \centering
    \begin{tikzpicture}[->,>=stealth',shorten >=1pt,auto,node distance=3cm,
                        thick, node/.style={circle,draw}, main node/.style={draw}]
    
      \node[main node] (1) {0};
      \node[node] (2) [above of=1] {1};
      \node[node] (3) [below of=1] {2};
      \node[node] (4) [right of=1] {3};
      \node[node] (5) [above right of=4] {4};
      \node[node] (6) [left of=5] {5};
      \node[node] (7) [below of=5] {6};
      \node[node] (8) [left of=1] {8};
      \node[node] (9) [right of=3] {7};
      
      \extralabel  {45}{\((6; 7)\)}{2};
      \extralabel  [1mm]{90}{\((5; 5)\)}{4};
      \extralabel  [1mm]{90}{\((3; 7)\)}{5};
      \extralabel  {90}{\((15; 11)\)}{6};
      \extralabel  {315}{\((18; 12)\)}{7};
      \extralabel  [1mm]{0}{\((10; 8)\)}{9};
      \extralabel  {315}{\((5; 6)\)}{3};
      \extralabel  {135}{\((4; 10)\)}{8};
    
      \path[every node/.style={}]
        (1) edge node [left] {0} (9)
        (9) edge node [left] {2} (7)
        (7) edge node [left] {8} (5)
        (5) edge node [left] {4} (6)
        (6) edge node [left] {8} (3)
        (3) edge node [left] {7} (8)
        (8) edge node [left] {1} (2)
        (2) edge node [left] {0} (1);
        
    \end{tikzpicture}
    \caption{Exemplo de uma rota viável \citep{art:REF_ART_2}}
    \label{fig:my_label}
\end{figure}